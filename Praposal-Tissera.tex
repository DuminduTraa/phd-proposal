\documentclass[a4paper,oneside,12pt]{report}
\usepackage{graphicx}
\usepackage{titlesec}
\usepackage{cite}
\usepackage{setspace}
\usepackage{amsmath}
\usepackage{amssymb}
\usepackage{refstyle}
\usepackage[titletoc]{appendix}

%\usepackage{tocloft}
%\usepackage[Lenny]{fncychap}

\usepackage[a4paper,bindingoffset=0.0mm,%
left=40mm,right=25mm,top=25mm,bottom=40mm,%
footskip=.25in]{geometry}
\usepackage{blindtext}


\renewcommand\contentsname{Table of Contents}

%\renewcommand{\cftchappresnum}{Chapter }  
%\cftsetindents{chapter}{0em}{6em}  
%\addtocontents{toc}{~\hfill\textbf{Page}\par}       % add 'Page' to ToC


\renewcommand{\baselinestretch}{1} 

\titleformat{\chapter}[display]
{\filleft\normalfont}{\scshape\chaptertitlename\ \thechapter}{-20pt}{\large\bfseries\MakeUppercase}
[\vspace{-3.5ex}
\rule{\textwidth}{0.3pt}]
\titlespacing{\chapter}
{0pt}{10pt}{20pt}
\renewcommand{\chaptername}{Section}


\titleformat{\section}[hang]
{\normalfont\bfseries}{\thesection}{0.5em}{}
\titlespacing{\section}
{0pt}{20pt}{20pt}

\titleformat{\subsection}[hang]
{\normalfont\bfseries}{\thesubsection}{0.5em}{}
\titlespacing{\subsection}
{0pt}{10pt}{10pt}

\titleformat{\subsubsection}[hang]
{\normalfont\bfseries}{\thesubsubsection}{0.5em}{}
\titlespacing{\subsubsection}
{0pt}{10pt}{10pt}


\begin{document}
	
%\title{A Sample Thesis}
%\vspace{10mm}
%\author{Dannangoda Gamage Kanishka Madusanka}
%\vspace{10mm}
%{(148039D)}
%\vspace{10mm}
%\date{January 2009}
%\maketitle


\begin{titlepage}
	\begin{center}
		\vspace*{5cm}
		
		\huge
		\textbf{Vision Based Navigation for Aerial and Ground Vehicles }
		
		\vspace{0.5cm}
		
		\large
		\vspace{0.5cm}
		
		M.H.G.D. Tissera
		
		\vspace{0.5cm}
		
		(XXXXXXX)
		\vfill
		
		\normalsize
		Research Proposal 
		
		\vspace{0.8cm}
		\large
		
		Department Electronic and Telecommunication Engineering\\
		\vspace{0.8cm}
		University of Moratuwa\\
		Sri Lanka\\
		\vspace{0.8cm}
		June 2018
		
	\end{center}
\end{titlepage}

\pagenumbering{roman}

\setlength{\parskip}{1em}
\setstretch{1.5}

% Declaration

%\chapter*{Declaration}
%\addcontentsline{toc}{chapter}{Declaration}
\begin{center}
\huge
\textbf{UNIVERSITY OF MORATUWA}
\end{center}

\begin{flushleft}
\large
\textbf{Declaration}
\end{flushleft}
\setlength{\parindent}{0cm}

I declare that this is my own research proposal and this proposal does not incorporate without acknowledgment any material previously published submitted for a Degree or Diploma in any other university or institute of higher learning and to the best of my knowledge and belief it does not contain any material previously published or written by another person except where the acknowledgment is made in the text.

Signature:  \hspace{7cm}  Date:\\
\\
.................................... \hspace{4.25cm} ....................\\
M.H.G.D. Tissera

I have read the proposal and it is in accordance with the approved university proposal outline. I am willing to supervise the research work of the above candidate on the proposed area.

Signature of the Supervisor(s):  \hspace{3.25cm}  Date:\\
\\
.................................... \hspace{4.25cm} ....................\\
Dr.Ranga Rodrigo\\
\\
.................................... \hspace{4.25cm} ....................\\
Dr.Beshan Kulapala\\
\\


\clearpage


% abstract
\begin{flushleft}
	\large
	\textbf{Abstract}
\end{flushleft}
\addcontentsline{toc}{chapter}{Abstract}
\setstretch{1.5}
\small

To be filled 

\cite{R01} \cite{R02} \cite{R03} \cite{R04} \cite{R05} \cite{R06} \cite{R07} \cite{R08} \cite{R09} \cite{R10} \cite{R11} \cite{R12} \cite{R13} \cite{R14} \cite{R15} \cite{R16} \cite{R17} \cite{R18} \cite{R19} \cite{R20} \cite{R21} \cite{R22} \cite{R23} \cite{R24} \cite{R25} \cite{R26} \cite{R27} \cite{R28} \cite{R29} \cite{R30} \cite{R31} \cite{R32} \cite{R33} \cite{R34} \cite{R35} \cite{R36} \cite{R37} \cite{R38} \cite{R39} \cite{R40} \cite{R41} \cite{R42} \cite{R43} \cite{R44} \cite{R45} \cite{R46} \cite{R47} \cite{R48}
\cite{R49} \cite{R50} \cite{R51} \cite{R52} \cite{R53} \cite{R54} \cite{R55} \cite{R56} \cite{R57} \cite{R58} \cite{R59} \cite{R60}
\cite{R61} \cite{R62} \cite{R63} \cite{R64} \cite{R65} \cite{R66}


\textbf{\textit{Keywords:} XXXXXXXXXXXX}



\setlength{\parindent}{5mm}
\normalsize
\setstretch{0.8}

\tableofcontents
\addcontentsline{toc}{chapter}{Table of Contents}
\clearpage


\clearpage

\pagenumbering{arabic}

% --------------------------------------------------------------------------------------------------------------------------------------------------------------------------

\setlength{\parindent}{5mm}
\normalsize
\setstretch{1.5}
\chapter{Introduction}
\label{ch:introduction}

Introducing autonomy for human made machines, is a highly demanding and challenging area in the field of Engineering. Applications of this area have been spread out a in wide range starting from sophisticated auto pilot system of an air plane to the service robots utilized by the general public. Autonomous navigation of robots, have become one of key areas in the field of robotics and introducing intelligence and self-governing ability for the navigation, is still a huge challenge, because of the complex nature attached it. Therefore, various types of control architectures such as deliberative, ractive and hybrid control architectures have been proposed and tested since 1960's in order to achieve robust, flexible, reliable and high performance control systems for autonomous navigation of mobile robot \cite{R01}. An intelligent mobile robot platform should be capable of sensing and understanding the environment, planing path toward desired goal efficiently, adapting for the changes in the environment and platform and also making quick decisions in a dynamic environment. Control architectures defines the way of integrating those abilities in order to drive the platform towards the goal while achieving all desired outcomes \cite{R01} \cite{R61}.

Although various control architectures were devlopped to address problems in autonomous navigation, the platform dependency of a given solution can be identified as  one of the less addressed issue in the field of autonomous navigation \cite{R02}. Suppose, if a particular platform with a perfect algorithm was used to accomplish a certain complex task. However the same approach might not be suitable to perform the same task for another platform which is even slightly different form the first one or the operating environment was changed scarcely. Whereas, if a kid was given two remote controlled toy cars, one is a differential-driven car and the other one is an omni-wheeled driven car and this kid might be able to move both cars from one point to another along a given path, after few training sessions even though he doesn't have any idea of mathematical models of those platforms. He could easily overcome the platform dependent learning by having few information about the both platforms and the experience gathered from training sessions. Also he is capable of reusing the experience he gained in practicing the differential-driven car, to learn the behavior of the another type of toy car and it reduces the training duration. A similar kind of intelligence can be given to an agent to navigate different robotic platforms, if there is generalized method of representing the experiences and  identifying common features in autonomous navigation with respect to each platform.

The main objective of this research is to overcome the issues of platform dependency in  autonomous navigation problems by developing a common  architecture for learning the behavior of the platform by itself. Such a system will not be required to provide information about the platform behavior, environment and the complete mathematical model of the system extensively. Reinforcement learning (RL) framework provides a promissing and powerful lead \cite{R24} in this context, since it enables a robot to explore its optimal behavior by interacting with its environment using a trial and error approach, without detailing the problem explicitely \cite{R03}. Only a some knowledge of the system will be required to initialize the training process and eventually it will be able to learn it's behavior completely with the progress of training process.  Since this effort is to develop a common architecture, it is possible to transfer the knowledge and the experience to another platform effectively. Also a virtual environment can be used for initial training process \cite{R07} \cite{R08} \cite{R10} \cite{R24} of the agent and it avoid many problems arise while training the agent using trial and error approach in real world such as getting damaged some unstable and fragile systems like areal vehicle \cite{R10}. Also introducing the knowledge  transfer-ability to the agent will lead to enhance the overall efficiency of entire learning process\cite{R13}.





% --------------------------------------------------------------------------------------------------------------------------------------------------------------------------

\chapter{Literature Review}
\label{ch:litearure review}

\section{Autonomous Navigation Architectures}

Introducing autonomous navigation for various robotic platforms and unmanned vehicles is one of the major developing and interesting area in the field of robotics and Artificial Intelligence (AI). 

\section{Reinforcement Learning for Autonomous Navigation}
Most of the trainable adaptive intelligent control architectures of various robotic platforms have been designed based on supervised learning techniques such as Learning from Demonstration (LfD)\cite{R38}. In supervised learning problems the agent is directly trained by using a set of examples with correct predictions which is considered as a supervisor or teacher and it tries to generalize the solution for unseen circumstances \cite{R03}. Typically, the demonstrator for such control systems will be a linear or nonlinear control law, hand written control policy, or it may be a human expert \cite{R05} \cite{R23}.  However, in some cases the role of a demonstrator might not be feasible to implement using above techniques. In such a situation it may be might be more appropriate to explore the optimal control strategies by means of on-line methods like reinforcement learning since it derives control rules as functions of the states while evaluating the overall behavior of a platform \cite{R38}.
\subsection{Reinforcement Learning Framework}
Reinforcement Learning (RL) provides a powerful framework and a set of tools for robotics since it enables a robot to explore and learn to execute actions in a optimal way,  by interacting with the environment using a trial and error approach \cite{R03}. Typically a RL agent observes current state $s_t$ and performs an action $a_t$ at a particular time-step t while following a particular policy $\pi$. Thereafter the environment accepts that action and generates a new state $s_{t+1}$ and a  reward $r_{t+1}$. This reward is a scalar value which represents the feedback for the action played and it measures the performance of a single time step with respect to the action performed. Then the agent accepts the reward and take another action by observing the newly generates state and this process will be continued until the agent meets a termination state as illustrated in the. simply an RL agent is learning how to map situations to actions so that to maximize the expected numerical reward signals. Also the agent is not aware of taking the action in an optimal way, but instead it must discover a mechanism to select which actions to be played in order to maximize the sum of discounted future rewards, by following a trial and error approach \cite{R37}.
%\begin{figure}[h]
%\begin{center}
%  \includegraphics[scale = 0.5]{Figure01.png}
%  \caption{Reinforcement Learning Faramework.}
%  \label{fig:Figure01}
%\end{center}
%\end{figure}

\subsection{Reinforcement Learning in the Context of Robotics}

one of the major objectives of developing intelligent robotic platforms is to solve various complex tasks by in-cooperating continuous, high-dimensional, unprocessed, sensory input data and controling continuous, high-dimensional action-space \cite{R06}. When the domain of reinforcement learning is considered, the difficulty of applying RL techniques on continuous high-dimensional state and action spaces can be identified as one of the major disadvantages. In the context of autonomous navigation most of the navigation problems are described in terms of continuous and high dimensional state and action spaces or large discreate state action spaces. In robotics it is not possible to observe exact state of the robot since the states are not completely observable and noise-free \cite{R03}. Therefore the learner might not be able to identify its state precisely and some states with considerable difference will be recognized as similar states \cite{R03}. Thus, most of the reinforcement learning problems in robotics are usually modeled as partially observed \cite{R03}.

\subsection{Deep Reinforcement Learning}

In general the applicability of reinforcement learning on the field of robotics has limited to domains \cite{R40} \cite{R41} \cite{R42} where the useful features can be handcrafted or to problems with fully observed low-dimensional state spaces \cite{R04}. The main reason for the above issue is that it is impractical to visit or represent large state and action spaces (continuous or discretised) which can be found in most of the real-world robotics problems \cite{R03}. In this setting, the function approximation techniques can be integrated to reinforcement learning algorithms \cite{R04} \cite{R04} \cite{R06} \cite{R15} in order to generalize value functions based on samples collected while interacting with the environment \cite{R03}. 

Function approximation \cite{R43} is set of  mathematical and statistical methods of computing a function which can be used to represent a particular set of data exactly or explicitly. Many research groups have taken various efforts to apply Deep Neural Networks (DNN)\cite{R40} \cite{R44} \cite{R45} \cite{R46} \cite{R47} for reinforcement learning as  a global non-linear value-function approximator \cite{R03}.The most significant attribute of deep learning is that deep neural networks are capable of extracting compact low dimensional representations of high-dimensional data automatically. However, employing non-linear function appropriators such as neural networks to estimate action-value function (Q-function) reinforcement learning was known to be fundamantally unstable or even to diverge \cite{R04} \cite{R06} due to several reasons such as correlation present in observation data and correlation between action-values (Q-value) and target values \textit{etc}. 

Mnih \textit{et al.} \cite{R04} in 2015 developed a novel reinforcement learning agent known as deep Q-network (DQN) which can learn successful policies directly from high-dimensional state space. They ware able to run-over the above mentioned issues by introducing the concept of experience replay \cite{R48} \cite{R49} and by using iterative update for the action-value function and updating the target function periodically. This experiment can be recognized as a major breakthrough in the reinforcement setting where the deep neural networks were employed successfully in implementation of reinforcement learning algorithms to address the difficulties in handling large high-dimensional state space. These discoveries guide us with a new approach to penetrate the blockade that we had in addressing real-world complex robotics tasks like navigation problems, with reinforcement learning methods.

Among the resent successful stories in the field of deep reinforcement learning (DRL), the DQN agent developed by Mnih \textit{et al.} \cite{R04} can be considered as the kick-starting revolution of DRL \cite{R50} which could learn to play a range of classical Atari 2600 video games directly from raw image pixels, at a level of comparable to professional human players. Another outstanding effort in the field of DRL is the development of hybrid DRL agent \cite{R52} (a combination of supervised learning and reinforcement learning algorithms) known as AlphaGo \cite{R51} which could manage to defeat a world champion in the game of GO \cite{R50}.However these advancements have motivated to apply DRL algorithms to wide range of robotics problems such as learning control strategies from camera inputs \cite{R53} \cite{R54} and developing adaptive robotics agents to perform various complex navigation tasks \cite{R08} \cite{R18} \cite{R23} \cite{R29} \textit{etc.}
 
\subsection{Reinforcement Learning with Continuous state and action spaces }

One of the primary goal of robotics is to formulate adaptable, intelligent robotics systems which can solve sophisticated real world tasks by perceiving the environment with the support of unprocessed, high-dimensional sensory data \cite{R06}. The DQN algorithm proposed by Mnih \textit{et al.} \cite{R04} is capable of handling problems with high-dimensional observation space in both continuous or discrete domains very effectively and efficiently. However one limitation of the DQN algorithm is it can only admit to problems with low-dimensional, discrete action space \cite{R06} \cite{R19}\cite{R50}. Yet most of the real world physical control problems have formulated with continuous high-dimensional action domains where the DQN algorithms cannot be directly applied. The main reason for this shortcoming is that the DQN algorithm highly relies on finding the action corresponding to the highest action-value at a given instant (see the appendix A), which is computationally expensive or unrealistic to calculate in most of the cases. On the other hand discretization of actions will be sufficient only for problems with low number of degree of freedoms and low number of discrete steps. Otherwise it creates many complications in a setting where a fine control of actions are demanding. In addition to that it limits the ability to explore the whole action space efficiently.


The advancement in fields of Deep Learning (DL) and Reinforcement Learning (RL) has dramatically improved the perforce of DQN algorithms while overcoming above described ingrained weaknesses found in deep Q-network. Thimothy \textit{et al.} \cite{R06} has proposed a model-free, off-policy actor-critic algorithm termed as Deep Deterministic Policy Gradient (DDPG) algorithm (see the appendix B), by employing deep neural network as function appropriator that can learn competitive policies in high-dimensional continuous action space. Their approach is based on Deterministic Policy Gradient (DPG) algorithm \cite{R14} and it is a combination of actor-critic approach\cite{R55} and outstanding features of DQN \cite{R04} \cite{R06}. 

In 2016 Mnih \textit{et al.} introduced a new paradigm for the reinforcement learning framework known as Asynchronous Deep Reinforcement Learning (ADRL)\cite{R20}.Rather than applying experience replay as proposed in DQN, in asynchronous methods multiple agent have been executed parallelly and asynchronously. Proposed asynchronous algorithms have exhibited better results in less time using less resources compare to distributed methods \cite{R56} and the asynchronous advantage actor-critic (A3C) algorithm (see appendix E) was the mostly outperformed algorithm compare to other suggested methods (see appendix C and D). In 2017 Lei  \textit{et al.} \cite{R07} fused asynchronous RL methods with  deep deterministic policy gradient method \cite{R57} \cite{R58} \cite{R59} and developed a novel algorithm called Asynchronous Deep Deterministic Policy Gradient(ADDPG) and applied it to develop mapless navigation system in continuous domain. These asynchronous methods can be used to train RL agents to solve complex problems with continuous observation and action spaces, effectively and efficiently in terms of training time and resources.

\subsection{XXXXXXXXX}

To be filled


\section{Knowledge Transfer Methods in Robotics}

\subsection{XXXXXXXXXXX}

To be filled

\subsection{XXXXXXXXXXXX}

To be filled
% --------------------------------------------------------------------------------------------------------------------------------------------------------------------------

\chapter{Research Objectives}
\label{ch:research objectives}

The main objective of this research is to develop a common architecture which facilitate to learn the behavior of various robotics platforms to overcome the platform dependency problem and incorporate it for autonomous navigation of various robotics platforms. 

Research Objectives:

\begin{enumerate}
%1	
\item To Develop a common architecture to overcome the platform dependency problem when performing high-level navigation tasks

%2
\item To create an agent which is operating on the proposed architecture with the capable of learning the platform behavior 

%3
\item To utilize the developed architecture for autonomous navigation of different robotics platforms 

%5
\item To develop and implement efficient, effective and safe methods to train the agent as appropriate with autonomous navigation.  

%6
\item To evaluate the performance of proposed architecture by training the agent to perform some high-level navigation task and transferring the knowledge gained by one platform to a similar kind of another platform 




\end{enumerate}

% --------------------------------------------------------------------------------------------------------------------------------------------------------------------------

\chapter{Research Methodology}
%\vspace{-5mm}
\label{ch:research methodology}

\begin{enumerate}
%1	
\item Carrying out comprehensive literature review as covering following areas to develop a suitable architecture for autonomous navigation of mobile robots.
	\begin{itemize}
		\item Conventional autonomous navigation methods for mobile robot
		\item Artificial Intelligence (AI)
		\item Artificial Neural Networks (ANN)\cite{R09}
		\item Cognitive architectures
		\item Memory architectures for autonomous navigation
		\item Deep Reinforcement Learning (DRL) in robotics 
		\item Knowledge transfer methods in robotics

	\end{itemize}
	
%2
\item Implementing an adaptable Deep Reinforcement Learning (DRL) based virtual robot agent which is capable of performing different navigation tasks such as path planing,target reaching and obstacle avoidance \textit{etc}, and simulating it in virtual Robotic Experiment Platform (V-rep)\cite{R60}. The purpose of implementing this agent is to get a better understanding of training based navigation problems and to study about various training methods.

%3
\item Studying about Knowledge transferring and incorporating the existing knowledge to develop efficient and safe training methods.

%4
\item Studying about existing cognitive architectures and developing a common architecture to overcome the platform dependency problem.

%5
\item Developing an agent to run on the developed architecture and evaluate the performance of it using two quad-rotor platforms.

%6
\item The successful operation of proposed architecture will be demonstrated using two quad-rotor platforms with a vision system. One of the quad-rotor will be controlled manually and that will be used to train the second quad-rotor. The second quad-rotor should be able to learn the platform behavior by itself and follow the movement of the manually controlled quad-rotor

%7
\item To demonstrate the platform independent learning ability, some of the controlled parameters of second quad rotor will be changed after the first training and the ability of mimic the movement of  manually controlled quad-rotor with the same performance, will be evaluated.



\end{enumerate}
% --------------------------------------------------------------------------------------------------------------------------------------------------------------------------

\chapter{Work Plan and Resource Requirements}
\label{ch:work plan and resource requirements}

% Image of work plan
%\begin{figure}[h]
%\begin{center}
  %\includegraphics[scale = 0.45]{work_plan.png}\\
  %\caption{Work Plan.}\label{fi:work_plan}
%\end{center}
%\end{figure}

Resource Requirement

\begin{enumerate}
\item Two quad-rotors with a vision system and remote control capability.
\item Two high performance single board computers

\end{enumerate}

% --------------------------------------------------------------------------------------------------------------------------------------------------------------------------

\chapter{Conclusion}
\label{ch:conlusion}

To be filled 

%-----------------------------------------------------------------------------------------------------------------------------------------------------------------

\renewcommand{\bibname}{\normalfont\selectfont\normalsize References}
\bibliographystyle{IEEEtran}
\bibliography{reference}
\addcontentsline{toc}{chapter}{References}
\renewcommand{\bibname}{whatever}

\begin{appendices}
\chapter{Deep Q-Learning Algorithm with Experience Replay}

Deep Q-Learning algorithm with experience replay technique as proposed by Mnih \textit{et al.} in 2015.

\chapter{Deep Deterministic Policy Gradient Algorithm}

Deep Deterministic Policy Gradient (DDPG) Algorithm proposed by Thimothy \textit{et al.} in 2016.

\chapter{Asynchronous One-Step Q-Learning Algorithm}

\chapter{Asynchronous n-Step Q-Learning Algorithm}

\chapter{Asynchronous Advantage Actor-Critic Algorithm}
\end{appendices}

\end{document}