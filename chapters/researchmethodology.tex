
\begin{enumerate}
%1	
\item Carrying out comprehensive literature review as covering following areas to develop a suitable architecture for autonomous navigation of mobile robots.
	\begin{itemize}
		\item Conventional autonomous navigation methods for mobile robot
		\item Artificial Intelligence (AI)
		\item Artificial Neural Networks (ANN)\cite{R09}
		\item Cognitive architectures
		\item Memory architectures for autonomous navigation
		\item Deep Reinforcement Learning (DRL) in robotics 
		\item Knowledge transfer methods in robotics

	\end{itemize}
	
%2
\item Implementing an adaptable Deep Reinforcement Learning (DRL) based virtual robot agent which is capable of performing different navigation tasks such as path planing,target reaching and obstacle avoidance \textit{etc}, and simulating it in virtual Robotic Experiment Platform (V-rep)\cite{R60}. The purpose of implementing this agent is to get a better understanding of training based navigation problems and to study about various training methods.

%3
\item Studying about Knowledge transferring and incorporating the existing knowledge to develop efficient and safe training methods.

%4
\item Studying about existing cognitive architectures and developing a common architecture to overcome the platform dependency problem.

%5
\item Developing an agent to run on the developed architecture and evaluate the performance of it using two quad-rotor platforms.

%6
\item The successful operation of proposed architecture will be demonstrated using two quad-rotor platforms with a vision system. One of the quad-rotor will be controlled manually and that will be used to train the second quad-rotor. The second quad-rotor should be able to learn the platform behavior by itself and follow the movement of the manually controlled quad-rotor

%7
\item To demonstrate the platform independent learning ability, some of the controlled parameters of second quad rotor will be changed after the first training and the ability of mimic the movement of  manually controlled quad-rotor with the same performance, will be evaluated.



\end{enumerate}